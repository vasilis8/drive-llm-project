% ============================================================
% Language-Conditioned Racing Agent — Semester Project Report
% ============================================================
\documentclass[12pt,a4paper]{article}

% ---- Packages ----
\usepackage[utf8]{inputenc}
\usepackage[T1]{fontenc}
\usepackage{lmodern}
\usepackage[margin=2.5cm]{geometry}
\usepackage{amsmath,amssymb}
\usepackage{graphicx}
\usepackage{booktabs}
\usepackage{hyperref}
\usepackage{caption}
\usepackage{subcaption}
\usepackage{xcolor}
\usepackage{listings}
\usepackage[backend=biber,style=numeric,sorting=none]{biblatex}
% \addbibresource{references.bib}  % Uncomment when you have references

\hypersetup{
    colorlinks=true,
    linkcolor=blue!60!black,
    citecolor=blue!60!black,
    urlcolor=blue!60!black,
}

\lstset{
    basicstyle=\ttfamily\small,
    backgroundcolor=\color{gray!10},
    frame=single,
    framerule=0pt,
    breaklines=true,
    columns=fullflexible,
}

% ---- Title ----
\title{%
    \textbf{Language-Conditioned Racing Agent}\\[0.5em]
    \large Mapping Natural Language Commands to Continuous Vehicle Control\\[0.3em]
    \normalsize Semester Project — Autonomous Agents
}
\author{
    Vasilis K.\\
    \texttt{vasilis@example.com}  % TODO: Replace with your email
}
\date{February 2026}

% ============================================================
\begin{document}
\maketitle

% ---- Abstract ----
\begin{abstract}
This report presents the design, implementation, and evaluation of a
\emph{Language-Conditioned Racing Agent} that maps natural language commands
(e.g., ``Push for overtake'', ``Conserve tires'') and raw 2D LiDAR
scans to continuous steering and velocity outputs. The architecture
combines a pre-trained Transformer-based instruction encoder with a
LiDAR perception network, fused into a Proximal Policy Optimization
(PPO) policy trained in the CARLA simulator. We demonstrate that the
agent can adapt its driving behaviour in response to different
high-level strategic commands.

% TODO: Update with actual results once training is complete.
\end{abstract}

\tableofcontents
\newpage

% ============================================================
\section{Introduction}
\label{sec:intro}

In high-performance autonomous racing, agents typically optimise a
static objective such as minimum lap time. However, dynamic race
scenarios require adaptable strategies triggered by high-level
instructions—for example, \emph{``Defend the inside line''},
\emph{``Push for overtake''}, or \emph{``Conserve tires''}. This
project proposes and develops a \textbf{Language-Conditioned Racing
Agent} that learns to interpret such commands and adjust its driving
policy accordingly.

The key research question is: \emph{Can we train a reinforcement
learning agent that modulates its racing strategy in real-time based on
natural language instructions, while maintaining safe and performant
vehicle control?}

% ============================================================
\section{Related Work}
\label{sec:related}

% TODO: Add related work references
\subsection{Autonomous Racing}
% Discuss F1TENTH, autonomous racing approaches, etc.

\subsection{Language-Conditioned Reinforcement Learning}
% Discuss works on language-grounded RL, instruction following, etc.

\subsection{Multimodal Perception for Driving}
% Discuss LiDAR-based perception, sensor fusion approaches, etc.

% ============================================================
\section{Approach}
\label{sec:approach}

\subsection{System Overview}
The proposed architecture consists of three main components:

\begin{enumerate}
    \item \textbf{Instruction Encoder:} A pre-trained Transformer model
        (\texttt{all-MiniLM-L6-v2}) that encodes natural language
        commands into a fixed-size semantic embedding.
    \item \textbf{Perception Network:} An MLP / 1D-CNN that processes
        raw LiDAR range data to extract track boundary and obstacle
        features.
    \item \textbf{PPO Policy:} A reinforcement learning policy that
        takes the concatenated state
        $[\mathbf{z}_{\text{lidar}} \oplus \mathbf{z}_{\text{cmd}}]$
        and outputs continuous control actions: steering angle
        $\delta$ and target velocity $v$.
\end{enumerate}

\subsection{Instruction Encoder}
% Detail the text encoder architecture

\subsection{Perception Network}
% Detail the LiDAR processing network

\subsection{Policy Network}
% Detail the PPO setup, action space, etc.

\subsection{Reward Design}
% Detail the reward function, including command-specific terms

% ============================================================
\section{Experimental Setup}
\label{sec:experiments}

\subsection{Simulation Environment}
We use CARLA as the primary simulation environment, configured with
a 2D LiDAR sensor and a 1:10 scale vehicle model.

\subsection{Training Configuration}
% Detail hyperparameters, training schedule, compute resources (RunPod)

\subsection{Commands}
The following commands are used during training and evaluation:

\begin{table}[h]
\centering
\caption{Training commands and their expected driving behaviours.}
\label{tab:commands}
\begin{tabular}{@{}lp{8cm}@{}}
\toprule
\textbf{Command} & \textbf{Expected Behaviour} \\
\midrule
``Push for overtake''   & Aggressive braking, apex clipping, maximum exit speed \\
``Conserve tires''      & Smooth trajectories, moderate speed, minimal lateral g-force \\
``Defend the inside''   & Prioritise inside-line positioning, block overtake attempts \\
% TODO: Add more commands as defined
\bottomrule
\end{tabular}
\end{table}

% ============================================================
\section{Results}
\label{sec:results}

% TODO: Fill in with actual results after training

\subsection{Quantitative Evaluation}
% Tables / plots: lap time, command adherence metrics, collision rate

\subsection{Qualitative Analysis}
% Trajectory visualisations, behaviour differences per command

\subsection{Ablation Studies}
% Effect of removing instruction encoder, different embedding models, etc.

% ============================================================
\section{Discussion}
\label{sec:discussion}

% TODO: Discuss findings, limitations, and future directions

% ============================================================
\section{Conclusion}
\label{sec:conclusion}

% TODO: Summarise contributions and key findings

% ============================================================
% \printbibliography  % Uncomment when references.bib is ready

\end{document}
